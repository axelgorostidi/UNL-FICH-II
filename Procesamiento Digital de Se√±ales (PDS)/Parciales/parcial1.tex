\documentclass[a4paper,10pt,spanish]{article}

% Preámbulo - Parte A

\usepackage[utf8]{inputenc} % Soporte para los acentos
\usepackage[T1]{fontenc}

\usepackage[spanish]{babel} % Capítulos, seciones, etc. en español

\usepackage[margin=2cm]{geometry} % Diseño del documento

\usepackage{multicol} % Escribir doble columna

\usepackage{xcolor} % Usar colores
\usepackage{pstricks}

\usepackage{enumerate} % Cambiar etiquetas de numeración
\usepackage[shortlabels]{enumitem} % Manejo adicional de etiquetas de numeración

\usepackage{graphicx} % Manejo de gráficos y figuras

\usepackage{makeidx} % Índice alfabético

% Paquetes adicionales de símbolos matemáticos
\usepackage{amsmath,amssymb,amsfonts,latexsym,cancel} 

% \usepackage{pslatex} % Fuente Times
% \usepackage{mathpazo} % Fuente Palatino
% \usepackage{mathptmx} % Fuente Times
% \usepackage{bookman} % Fuente Bookman
% \usepackage{newcent} % Fuente New Century Schoolbook
% \usepackage{helvet} % Fuente Helvetica
\usepackage{palatino} % Fuente Palatino
% \usepackage{pxfonts} % Fuente 
% \usepackage{txfonts} % Fuente
% \usepackage{concrete} % Fuente
% \usepackage{cmbright} % Fuente
% \usepackage{fourier} % Fuente

\usepackage{booktabs} % Opciones adicionales para el entorno tabular
\usepackage{longtable} % Para tablas de más de una página

\usepackage{tikz} % Creación de gráficos

% Preámbulo - Parte B

\pagestyle{myheadings} % Numeración de página en la parte superior

\title{\Huge Procesamiento Digital de Señales\\
Parcial 1}
\author{Darién Julián Ramírez}
\date{\vspace{-5ex}}

\begin{document}

\maketitle %Crea la página de título

\section{Introducción a Señales}

\begin{enumerate}
\item Desarrolle la clasificación fenomenológica para las señales dando un ejemplo de señal de cada clase.

\item Describa el procedimiento mediante el cuál podría verificar la estacionariedad de una señal aleatoria. ¿Qué parámetros estadísticos utilizaría en el proceso?

\item Defina estacionariedad y ergodicidad ¿Cuál es la ventaja práctica que brinda el conocimiento \textit{a priori} de que una señal es ergódica?

\item Defina ruido, relación señal ruido (SNR), fuentes de ruido y ejemplifique.

\item Defina transformaciones de rango y transformaciones de dominio. Ejemplifique para cada caso.

\item Describa las etapas de un conversor de señales analógicas a digitales (A/D).
\end{enumerate}

\section{Introducción a Sistemas}

\begin{enumerate}
\item Defina y ejemplifique:
	\begin{enumerate}
	\item Sistema de Respuesta Finita al Impulso (FIR - MA).
	\item Sistema AutoRegresivo (AR).
	\item Sistema causal (no anticipativo) y no causal (anticipativo).
	\item Sistema Lineal e Invariante en el Tiempo (LTI).
	\end{enumerate}
	
\item ¿Cuáles son los tipos en los que se clasifican los sistemas LTI de tiempo discreto desde el punto de vista de su respuesta al impulso? Defina cada tipo y escriba la ecuación correspondiente.

\item  ¿Cuáles son los tipos en que se dividen los sistemas IIR? Ejemplifique con señales genéricas en $\mathbb{R}^{4}$.

\item Los sistemas de la forma $y[n]=ax[n]+b$. ¿Son sistemas lineales? ¿Que características presentan?

\item Dé una clasificación de sistemas discretos utilizando como criterio las distintas propiedades que presentan. Explique brevemente cada una de las clases.

\item  Defina un sistema LTI de tiempo discreto en base a sus propiedades. Escriba la ecuación genérica que describe su comportamiento.

\item Partiendo de las propiedades de un sistema discreto lineal e invariante en el tiempo (LTI), obtenga la expresión general de la salida $y[n]$ a una entrada $x[n]$ cualquiera.

\item Un sistema lineal tiene una respuesta al impulso que varía con el tiempo en una forma conocida, de modo que usted sabe que la respuesta a una entrada impulsiva $\delta[n-k]$ es $h_{k}[n]$:
	\begin{enumerate}
	\item Escriba una expresión que permita encontrar la salida de este sistema para 		cualquier entrada $x[n]$.
	\item ¿Qué condición deben cumplir las $h_{k}[n]$ para que el sistema sea causal?
	\end{enumerate}

\end{enumerate}

\section{Convolución Discreta}

\begin{enumerate}
\item Explique con un ejemplo a partir de dos señales $x[n]$ y $h[n]\in\mathbb{R}^{4}$ la operación para calcular la convolución lineal mediante la denominada \textit{multiplicación término a término.}

\item Explique cómo realizaría las operaciones de convolución lineal y convolución circular discretas utilizando multiplicación matricial. Indique claramente la forma que tendría la correspondiente matriz de convolución en cada caso.

\item  Exprese la sumatoria de convolución y explique las propiedades de los sistemas LTI que están implícitas en la misma. ¿Cuál es la importancia de la operación de convolución?

\item Demuestre que la salida $y[n]$ de un sistema LTI, ante cualquier entrada, es la convolución lineal discreta entre la entrada $x[n]$ y la respuesta al impulso $h[n]$ del sistema. Identifique claramente los pasos en los que se aplican las propiedades de este tipo de sistemas.\\
Obtenga la salida ante cualquier entrada, de un sistema lineal cuya respuesta varía en el tiempo de una forma conocida.

\item Explique como utilizaría la Transformada Discreta de Fourier (TDF) para calcular la convolución lineal (discreta).
\end{enumerate}

\section{Espacios de Señales}

\begin{enumerate}

\item Defina y explique distancia euclidiana y muestre que cumple con las propiedades para ser una métrica.

\item ¿Cuál es la relación de las normas p y las medidas físicas de acción, energía, potencia, raíz del valor cuadrático medio y amplitud?

\item Defina distancia y norma, indicando similitudes y diferencias entre ambos conceptos.

\item Enumere y defina las propiedades que debe cumplir una función para ser considerada una métrica.

\item  ¿Cuál es la definición de la cuasi-norma denominada \textit{norma 0} y en qué tipo de aplicaciones se utiliza?

\item Defina y explique espacio métrico de las señales de potencia media finita.

\item El espacio de las \textit{señales continuas que tienen amplitud finita} se denota como $L_{\infty}(\mathbb{R})$. Identifique una expresión coloquial que pueda asociar al espacio denotado como $L_{2}(\mathbb{R})$. Defina matemáticamente los conjuntos de señales correspondientes y explique por qué constituyen espacios y de qué tipo.

\item  Demuestre que la proyección ortogonal minimiza el error de una aproximación mediante la combinación lineal de un conjunto de vectores en $\mathbb{R}^{n}$.

\item ¿Puede una base contener vectores l.i. pero no ortogonales? Si su respuesta es afirmativa indique qué inconvenientes tendría la base. Si su respuesta es negativa proponga dos contraejemplos.

\end{enumerate}

\section{Transformada Discreta de Fourier}

\begin{enumerate}
\item A partir de la señal $x(t)=2\sin(2\pi 12t)+\sin(2\pi 5t)$ se obtuvo la secuencia $x[n]$, de $2[seg.]$ de duración utilizando una frecuencia de muestreo $f_{m}=20 [Hz]$:
	\begin{enumerate}
	\item Indique cuál es la resolución frecuencial en la Transformada Discreta de 			Fourier (TDF) de $x[n]$.

	\item Grafique la magnitud de la TDF de $x[n]$. Justifique su respuesta.
	\end{enumerate}

\item Proponga un ejemplo de una señal temporal analógica que ha sido mal muestreada (con alias) y grafique su espectro aproximado indicando claramente cómo se manifiesta el fenómeno en ambos dominios (temporal y frecuencial).

\item Defina y explique la propiedad de desplazamiento frecuencial para la Transformada Discreta de Fourier (TDF).

\item Indique cuáles son las diferencias entre la Transformada de Fourier de Tiempo Discreto (TFTD) y la Transformada Discreta de Fourier (TDF).

\item Defina y ejemplifique dos propiedades de la TDF.

\item Esquematize y explique el fenómeno de alias desde el punto de vista temporal y frecuencial utilizando como ejemplo una señal $x(t) = \cos(2\pi 11t)$ que ha sido muestreada a $10[Hz]$.

\item Para una señal discreta $x[n]$ de 4 muestras, escriba la TDF en forma matricial, identifique los elementos de la matriz que toman valores iguales y explique como se relaciona esta propiedad con la Transformada Rápida de Fourier (TRF).

\item ¿Cuáles son las diferencias entre la Transformada Discreta de Fourier (TDF) y la Serie de Fourier?

\item ¿Qué tipo de distorsiones incorpora el uso de ventanas temporales rectangulares en el espectro de las señales resultantes? ¿Cómo puede minimizarse este efecto?

\item Explique y esquematice el fenómeno de alias. ¿Por qué es importante el teorema de muestreo de Nyquist?

\item Explique el principio de incertidumbre de Heisenberg.

\item Defina la propiedad de convolución para las Transformadas Continua (TF) y Discreta de Fourier (TDF) y explique sus diferencias.

\item Explique desde el punto de vista del espacio de señales la expresión del teorema  de Parseval aplicado a una secuencia $x[n]$, de $N$ muestras y su TDF $X[k]$.

\item ¿Cuál es la propiedad fundamental de la TDF que permite que se cumpla el teorema de Parseval? ¿Se cumple una propiedad similar para todas las normas-$p$?

\item A partir de la señal $x(t)=\sin(2\pi 100t)+2\sin(2\pi 50t)$ se obtuvo la secuencia $x[n]$ utilizando una frecuencia de muestreo $f_{m}=150[Hz]$ durante un segundo.·¿Cómo es el espectro de frecuencias de $x[n]$? Grafíquelo respetando las escalas adecuadas y justifique su respuesta.

\item Se tiene una señal muestreada a una determinada frecuencia de muestreo $f_{m}$, durante un tiempo dado $T$. Explique lo que cambia en la resolución frecuencial y la frecuencia máxima al realizar los siguientes cambios:
	\begin{enumerate}
	\item Se cambia la frecuencia de muestreo a $2f_{m}$ manteniendo $T$.
	\item Se mantiene $f_{m}$, se cambia el tiempo total a $2T$.
	\end{enumerate}

\item Considere la propiedad de convolución de la TDF. ¿Bajo qué condiciones se verifica esta propiedad?

\item Suponga que $x[n]$ es una versión correctamente muestreada de una señal continua $x(t)$:
	\begin{enumerate}
	\item Explique la relación que existe entre la Transformada Discreta de Fourier 		(TDF) de $x[n]$ y la Transformada de Fourier de Tiempo Discreto (TFTD) de $x[n]$.
	
	\item Explique la relación que existe entre la TDF de $x[n]$ y la Transformada de 		Fourier (TF) de $x(t)$.
	
	\item Proponga un procedimiento para recuperar $x(t)$ a partir de $x[n]$. ¿Es 			físicamente realizable esta operación? Justifique.
	\end{enumerate}

\end{enumerate}

\section{Transformada Z}

\begin{enumerate}
\item En un contexto de discretización de una señal continua $x(t)$, describa cuál es la relación entre la Transformada Z de $x[n]$ y la transformada de Laplace de $x(t)$.

\item ¿Por qué resulta necesario utilizar transformaciones conformes aproximadas para obtener la función de transferencia de un sistema discreto a partir de la transferencia de uno continuo?

\item Describa una de las transformaciones conformes mencionadas en el punto anterior, resaltando sus características principales y las consideraciones  prácticas a tener en cuenta para su empleo.

\item A partir de la ecuación general que define el comportamiento temporal de los sistemas lineales e invariantes en el tiempo (LTI) discreto obtenga la expresión general para su correspondiente función de transferencia.\\
Explique las diferencia para cada uno de los tres tipos de sistemas LTI discretos existentes.

\item Demuestre que la trasformada Z, es equivalente a la transformada de Laplace de una secuencia discreta $x[n]$.

\item ¿Cuál es el principal inconveniente que presenta la transformación bilineal para el mapeo entre los planos s y Z? Explique en forma gráfica cómo afecta la respuesta del sistema de tiempo discreto correspondiente.

\item Grafique en los planos $s$ y $z$ los mapeos realizados por la transformación de Euler, la bilineal y la ideal. En cada caso indique claramente todas las referencias en los ejes coordenados en función de la frecuencia de muestreo.

\item Explique los pasos para obtener la ecuación en diferencias de un sistema discreto a partir de la ecuación diferencial que define el sistema en tiempo continuo. ¿Qué propiedades de la Transformada Z deben aplicarse?

\item ¿Cuáles son las condiciones para que la Transformada Z de una secuencia converja?

\item Explique las principales desventajas de las transformaciones de Euler y Bilineal. Proponga además en cada caso un ejemplo donde se puedan apreciar las limitaciones de cada una.

\item Explique la relación entre la Transformada de Fourier de una secuencia discreta y la Transformada Z.

\item Se dispone de una señal $y(t)$ que se obtuvo como respuesta del sistema $H(s)=\frac{s+0.1}{(s+0.1)^{2}+16}$ a la entrada $x(t)=\sin(2\pi 10t)$. A partir de esta se obtuvo una señal $y[n]$ utilizando una frecuencia de muestreo $f_{m}=256[Hz]$ y $N=512$ muestras de duración. Además se sabe que el ancho de banda de $H(s)$ es de $20[Hz]$.\\
Considerando lo mencionado anteriormente conteste las siguientes preguntas:
	\begin{enumerate}
	\item Si $Y[k]$ es la TDF de $y[n]$, grafique las componentes distintas de cero de 	$Y[k]$ indicando el valor de la muestra respectiva y a qué frecuencia corresponde.
	\item ¿Cómo haría para obtener una estimación de las muestras intermedias de $y[n]		$ de tal manera que la longitud total sea de $4N$ muestras?
	\item ¿Cómo obtendría la respuesta en frecuencia $|H(\exp^{-j\theta})|$ del 			sistema discreto? Describa y justifique cada paso en el proceso.
	\end{enumerate}

\end{enumerate}

\end{document}