\documentclass{beamer}

% Preámbulo - Parte A

\usepackage[utf8]{inputenc} % Soporte para los acentos
\usepackage[T1]{fontenc}

\usepackage{xcolor} % Usar colores
\usepackage{pstricks}

% Paquetes adicionales de símbolos matemáticos
\usepackage{amsmath,amssymb,amsfonts,latexsym,cancel} 

\usepackage{booktabs} % Opciones adicionales para el entorno tabular
\usepackage{longtable} % Para tablas de más de una página

\usepackage{multirow}

\usepackage{listings} %Para escribir códigos
\lstset{language=XML,
	basicstyle=\footnotesize,
	numbers=left,
 	stepnumber=1,
	numbersep=8pt,
	showspaces=false,               % show spaces adding particular underscores
  	showstringspaces=false,         % underline spaces within strings
  	frame=lines,                   % adds a frame around the code
	tabsize=4,                      
  	captionpos=b,                   % sets the caption-position to bottom
  	breaklines=true,                % sets automatic line breaking
}
%---

\DeclareGraphicsExtensions{.pdf,.png,.jpg}
\usefonttheme{professionalfonts}
\usetheme{Warsaw}
\setbeamercovered{transparent}

\section{Inteligencia Computacional}

\begin{document}

\title{Búsqueda del camino óptimo entre dos ciudades mediante el algoritmo del sistema de hormigas}
\subtitle{Trabajo Final de Inteligencia Computacional}
\author{Marco A. Pereyra,
        Gaspar E. Oberti y 
        Darién J. Ramírez \\ Tutor: Molas Giménez, José Tomás}
\date{\empty}

\frame{\titlepage}

% 1
\begin{frame}{Introducción} %\frametitle{} \framesubtitle{}

\end{frame}
%%%%%%%%%%%%%%%%%%%%%%%%%%%%%%%%%%%%%%%%%%%%%%%%%%%%%%%%%%%%%%%%%%%%%%%%%%%%%%%%%%%%%
\begin{frame}[fragile]{algo}
\begin{columns}
\begin{column}{width=0.5\textwidth}
asdfasd
\end{column}
\begin{column}{width=0.5\textwidth}
    sadasd
\end{column}
\end{columns}
\end{frame}
%%%%%%%%%%%%%%%%%%%%%%%%%%%%%%%%%%%%%%%%%%%%%%%%%%%%%%%%%%%%%%%%%%%%%%%%%%%%%%%%%%%%%
\begin{frame}{Resultados}
\begin{tabular}[c]{cccc} \toprule
Dimensión de la matriz & Hormigas & Genético & Costo Uniforme \\ \midrule
$7\times7$    & 0.36  & 2.72 & 0.32     \\
$10\times10$  & 1.13  & 5.31 & 3.72     \\
$15\times15$  & 6.45  & 9.71 & 35.33    \\
$20\times20$  & 13.11 & 18.7 & 243.42   \\ \bottomrule    
\end{tabular}
\end{frame}
%%%%%%%%%%%%%%%%%%%%%%%%%%%%%%%%%%%%%%%%%%%%%%%%%%%%%%%%%%%%%%%%%%%%%%%%%%%%%%%%%%%%%
\begin{frame}{Resultados}
\begin{tabular}[c]{ccc} \toprule
Matrices involucradas & Hormigas & Genético\\ \midrule
Distancia       						& 70 & 62 \\
Distancia + Peaje     					& 62 & 58 \\
Distancia + Peaje + Hospedaje   		& 64 & 54 \\
Distancia + Peaje + Hospedaje + Tráfico & 60 & 50 \\ \bottomrule    
\end{tabular}
\vfill
Los tiempos de ejecución considerando sólo la distancia fueron de $17.03$ y $122.65$ segundos respectivamente.
\end{frame}
%%%%%%%%%%%%%%%%%%%%%%%%%%%%%%%%%%%%%%%%%%%%%%%%%%%%%%%%%%%%%%%%%%%%%%%%%%%%%%%%%%%%%
\begin{frame}{Resultados}
\begin{tabular}[c]{ccccc} \toprule
\multirow{2}{*}{Matrices involucradas}
& \multicolumn{2}{c}{7 Hormigas} 
& \multicolumn{2}{c}{14 Hormigas} \\
& $\rho=0.4$ & $\rho=0.8$ & $\rho=0.4$ & $\rho=0.8$ \\  \midrule
D             & 54 & 70 & 72 & 76 \\
D + P         & 62 & 76 & 80 & 88 \\
D + P + H     & 60 & 72 & 74 & 78 \\
D + P + H + T & 64 & 76 & 76 & 82 \\ \bottomrule    
\end{tabular}
\vfill
Tiempos: 18, 71, 36 y 112 segundos respectivamente.
\end{frame}
%%%%%%%%%%%%%%%%%%%%%%%%%%%%%%%%%%%%%%%%%%%%%%%%%%%%%%%%%%%%%%%%%%%%%%%%%%%%%%%%%%%%%
\begin{frame}{Conclusiones}
\begin{block}{}
\begin{enumerate}
\item \textbf{Costo uniforme:} Rápido para matrices pequeñas pero lento cuando se incrementa el tamaño.
\item \textbf{Algoritmo genético:} Tiempo de procesamiento intermedio. No depende de la representación. A veces se obtienen caminos con bucles.
\item \textbf{Sistema de hormigas:} Resultados medianamente buenos. Los tiempos de procesamientos están ligados al tamaño de la matriz, a la cantidad de hormigas y al $\rho$.
\end{enumerate}
\end{block}
\end{frame}
%%%%%%%%%%%%%%%%%%%%%%%%%%%%%%%%%%%%%%%%%%%%%%%%%%%%%%%%%%%%%%%%%%%%%%%%%%%%%%%%%%%%%
\begin{frame}{Conclusiones}
\begin{block}{}
\begin{enumerate}
\item Algo común tanto en el algoritmo de hormigas como en el genético es que a medida que incrementa el número de características de evaluación, los porcentajes de acierto del camino óptimo disminuyen.
\item Mejorar representación de los datos de entrada. Levantar datos de una base de datos de mapas, o procesar una imagen para detectar nodos y distancias entre ellos.
\end{enumerate}
\end{block}
\end{frame}
%%%%%%%%%%%%%%%%%%%%%%%%%%%%%%%%%%%%%%%%%%%%%%%%%%%%%%%%%%%%%%%%%%%%%%%%%%%%%%%%%%%%%
\begin{frame}{Bibliografía}
\begin{thebibliography}{99}
\bibitem{SRAT} %1
Marco Dorigo, Vittorio Maniezzo and Alberto Colorni,
\emph{The Ant System: Optimization by a colony of cooperating agents}. 
IEEE Transactions on Systems, Man, and Cybernetics?Part B, Vol.26, No.1, 1996, pp.1-13.

\bibitem{ID} %2
J. Aguilar, Member IEEE y M. A. Labrador, Senior Member IEEE
\emph{Un algoritmo de enrutamiento distribuido para redes de comunicación basado en sistemas de hormigas}. 
IEEE LATIN AMERICA TRANSACTIONS, VOL. 5, NO. 8,
DECEMBER 2007.
\end{thebibliography}
\end{frame}
%%%%%%%%%%%%%%%%%%%%%%%%%%%%%%%%%%%%%%%%%%%%%%%%%%%%%%%%%%%%%%%%%%%%%%%%%%%%%%%%%%%%%
\begin{frame}{Preguntas}

\begin{block}{}
\begin{center}
{\Huge ¿Preguntas?}
\end{center}
\end{block}

\end{frame}
%%%%%%%%%%%%%%%%%%%%%%%%%%%%%%%%%%%%%%%%%%%%%%%%%%%%%%%%%%%%%%%%%%%%%%%%%%%%%%%%%%%%%
\end{document}